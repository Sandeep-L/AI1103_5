\documentclass[journal,12pt,twocolumn]{IEEEtran}

\usepackage{setspace}
\usepackage{gensymb}
\singlespacing
\usepackage[cmex10]{amsmath}

\usepackage{amsthm}

\usepackage{mathrsfs}
\usepackage{txfonts}
\usepackage{stfloats}
\usepackage{bm}
\usepackage{cite}
\usepackage{cases}
\usepackage{subfig}

\usepackage{longtable}
\usepackage{multirow}

\usepackage{enumitem}
\usepackage{mathtools}
\usepackage{steinmetz}
\usepackage{tikz}
\usepackage{circuitikz}
\usepackage{verbatim}
\usepackage{tfrupee}
\usepackage[breaklinks=true]{hyperref}
\usepackage{graphicx}
\usepackage{tkz-euclide}

\usetikzlibrary{calc,math}
\usepackage{listings}
    \usepackage{color}                                            %%
    \usepackage{array}                                            %%
    \usepackage{longtable}                                        %%
    \usepackage{calc}                                             %%
    \usepackage{multirow}                                         %%
    \usepackage{hhline}                                           %%
    \usepackage{ifthen}                                           %%
    \usepackage{lscape}     
\usepackage{multicol}
\usepackage{chngcntr}

\DeclareMathOperator*{\Res}{Res}

\renewcommand\thesection{\arabic{section}}
\renewcommand\thesubsection{\thesection.\arabic{subsection}}
\renewcommand\thesubsubsection{\thesubsection.\arabic{subsubsection}}

\renewcommand\thesectiondis{\arabic{section}}
\renewcommand\thesubsectiondis{\thesectiondis.\arabic{subsection}}
\renewcommand\thesubsubsectiondis{\thesubsectiondis.\arabic{subsubsection}}


\hyphenation{op-tical net-works semi-conduc-tor}
\def\inputGnumericTable{}                                 %%

\lstset{
%language=C,
frame=single, 
breaklines=true,
columns=fullflexible
}
\begin{document}

\newcommand{\BEQA}{\begin{eqnarray}}
\newcommand{\EEQA}{\end{eqnarray}}
\newcommand{\define}{\stackrel{\triangle}{=}}
\bibliographystyle{IEEEtran}
\raggedbottom
\setlength{\parindent}{0pt}
\providecommand{\mbf}{\mathbf}
\providecommand{\pr}[1]{\ensuremath{\Pr\left(#1\right)}}
\providecommand{\qfunc}[1]{\ensuremath{Q\left(#1\right)}}
\providecommand{\sbrak}[1]{\ensuremath{{}\left[#1\right]}}
\providecommand{\lsbrak}[1]{\ensuremath{{}\left[#1\right.}}
\providecommand{\rsbrak}[1]{\ensuremath{{}\left.#1\right]}}
\providecommand{\brak}[1]{\ensuremath{\left(#1\right)}}
\providecommand{\lbrak}[1]{\ensuremath{\left(#1\right.}}
\providecommand{\rbrak}[1]{\ensuremath{\left.#1\right)}}
\providecommand{\cbrak}[1]{\ensuremath{\left\{#1\right\}}}
\providecommand{\lcbrak}[1]{\ensuremath{\left\{#1\right.}}
\providecommand{\rcbrak}[1]{\ensuremath{\left.#1\right\}}}
\theoremstyle{remark}
\newtheorem{rem}{Remark}
\newcommand{\sgn}{\mathop{\mathrm{sgn}}}
\providecommand{\abs}[1]{\vert#1\vert}
\providecommand{\res}[1]{\Res\displaylimits_{#1}} 
\providecommand{\norm}[1]{\lVert#1\rVert}
%\providecommand{\norm}[1]{\lVert#1\rVert}
\providecommand{\mtx}[1]{\mathbf{#1}}
\providecommand{\mean}[1]{E[ #1 ]}
\providecommand{\fourier}{\overset{\mathcal{F}}{ \rightleftharpoons}}
%\providecommand{\hilbert}{\overset{\mathcal{H}}{ \rightleftharpoons}}
\providecommand{\system}{\overset{\mathcal{H}}{ \longleftrightarrow}}
	%\newcommand{\solution}[2]{\textbf{Solution:}{#1}}
\newcommand{\solution}{\noindent \textbf{Solution: }}
\newcommand{\cosec}{\,\text{cosec}\,}
\providecommand{\dec}[2]{\ensuremath{\overset{#1}{\underset{#2}{\gtrless}}}}
\newcommand{\myvec}[1]{\ensuremath{\begin{pmatrix}#1\end{pmatrix}}}
\newcommand{\mydet}[1]{\ensuremath{\begin{vmatrix}#1\end{vmatrix}}}
\numberwithin{equation}{subsection}
\makeatletter
\@addtoreset{figure}{problem}
\makeatother
\let\StandardTheFigure\thefigure
\let\vec\mathbf
\renewcommand{\thefigure}{\theproblem}
\def\putbox#1#2#3{\makebox[0in][l]{\makebox[#1][l]{}\raisebox{\baselineskip}[0in][0in]{\raisebox{#2}[0in][0in]{#3}}}}
     \def\rightbox#1{\makebox[0in][r]{#1}}
     \def\centbox#1{\makebox[0in]{#1}}
     \def\topbox#1{\raisebox{-\baselineskip}[0in][0in]{#1}}
     \def\midbox#1{\raisebox{-0.5\baselineskip}[0in][0in]{#1}}
\vspace{3cm}
\title{AI1103 Assignment 5}
\author{Sandeep L -- CS20BTECH11044}
\maketitle
\newpage
\bigskip
\renewcommand{\thefigure}{\theenumi}
\renewcommand{\thetable}{\theenumi}

Download latex-tikz codes from 
%
\begin{lstlisting}
https://https://github.com/Sandeep-L/AI1103_5/blob/main/Assignment_5_AI1103.tex
\end{lstlisting}

\section*{Question 107}
Suppose $X$ follows an exponential distribution with parameter $\lambda>0$. Fix $a>0$. Define the random variable $Y$ by\\
$Y=k$, \qquad if $ka \leq X < \brak{k+1}a$,\\
$k=0,1,2\ldots$\\
Which of the following statements are correct?
\begin{enumerate}
\setlength\itemsep{0.5em}
    \item $\pr{4<Y<5}=0$
    \item Y follows an Exponential distribution
    \item Y follows a Geometric distribution
    \item Y follows a Poisson distribution
\end{enumerate}

\section*{Solution}

Since $Y$ takes only the value of positive integers defined by
\begin{align}
Y=
    \begin{cases}
        k & ka \leq X < \brak{k+1}a
    \end{cases}
\end{align}
for $k=0,1,2\ldots$ and $a>0$\\
So $Y$ doesn't take any value in \brak{4,5}.\\
Therefore, option \textbf{1)} $\pr{4<Y<5}=0$ is correct.\\

\newtheorem{def_X}{Definition}

\begin{def_X}
    $X$ follows an exponential distribution with parameter $\lambda>0$.Therefore, the P.D.F of X, i.e, $f_X\brak{x}$ is given by
    \begin{align}
    f_X\brak{x} =
        \begin{cases}
        \lambda e^{-\lambda x} & x\geq 0 \\
        0 & x<0
        \end{cases}
    \end{align}
\end{def_X}

Relation between $X$ and $Y$ for $k=0,1,2\ldots$ and $a>0$ is given by
\begin{align}
    Y=k \qquad ka\leq X < \brak{k+1}a
\end{align}

The P.M.F of $Y$ is given by
\begin{align}
    \pr{Y=k} &= \pr{ka\leq X < \brak{k+1}a}\\
    &= \int_{ka}^{\brak{k+1}a}f_X\brak{x}dx\\
    &= \int_{ka}^{\brak{k+1}a}\lambda e^{-\lambda x}dx\\
    &= \sbrak{-e^{-\lambda x}}_{ka}^{\brak{k+1}a}\\
    \pr{Y=k} &= e^{-a\lambda k}\brak{1 - e^{-a\lambda}}
\end{align}

Let $p=\brak{1-e^{-a\lambda}}$ in the above equation
\begin{align}
    \pr{Y=k} &= \brak{e^{-a\lambda}}^k\brak{1-e^{-a\lambda}}\\
    \pr{Y=k} &= \brak{1-\brak{1-e^{-a\lambda}}}^k\brak{1-e^{-a\lambda}}\\
    \pr{Y=k} &= \brak{1-p}^kp \qquad k=0,1,2\ldots
\end{align}

Therefore, option \textbf{3) $Y$ follows a Geometric distribution} is correct.

\end{document}